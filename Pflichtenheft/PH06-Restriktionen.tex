\section{Randbedingungen}


\vspace{1cm}
\begin{itemize}
    \item \textbf{/R010/}\textit{Internetzugang:} \par
    Das System darf zum Betrieb keine Internetverbindung benötigen. Dies ist nur während eines Updateprozesses zeitlich begrenzt möglich.
    
    \item \textbf{/R020/}\textit{Betriebssystem:} \par
    Als Betriebssystem ist eine Debian basierte Linux-Distribution vorgegeben.
    
    \item \textbf{/R030/}\textit{Hardware:} \par
    Als Server wird ein \textit{Raspberry Pi} vorgegeben. Als Client fungiert ein \textbf{Apple IPad} mit einer Bildschirmdiagonale von 11 Zoll. Der Ausdruck der Dokumente erfolgt durch einen Netzwerkdrucker oder einen über USB am Server angeschlossenen Drucker.
    
    \item \textbf{/R040/}\textit{Lokales Netzwerk:} \par
    Das System \textit{Raspberry Pi, IPad, Drucker} befinden sich nicht im WLAN der Filiale. Der Pi erzeugt zur Kommunikation ein eigenes Drahtlosnetzwerk.
    
    \item \textbf{/R050/}\textit{Frontend:} \par
    Durch die Nutzung eines \textit{IPads} als Client muss die Darstellung des Frontends auf dem Safari Browser möglich sein.
    
    \item \textbf{/R060/}\textit{Design:} \par 
    Das System soll auch für Menschen Nutzbar sein, welche sich nicht mit Technologie auskennen. Das Nutzerinterface ist daher simpel und deutlich lesbar zu gestalten.
\end{itemize}

