\section{Zielbestimmungen}

Der Grad der Digitalisierung in den deutschen Krankenkassen ist bereits weit fortgeschritten. So betreibt beispielsweise die AOK Plus ein Online Portal, mit den Funktionen eines digitalen Postfachs, der Änderung persönlicher Daten und dem Download von Anträgen und Formularen. Auf der anderen Seite existieren weiterhin Filialen der Krankenkassen vor Ort, die Service und persönliche Beratung bieten. Das Self-Service-Terminal (SST) stellt eine Verbindung von Vorort-Service und digitalen Angeboten dar. Es ergänzt das digitale Angebot der Krankenkassen um eine Möglichkeit der Selbstbedienung. Das SST soll insbesondere für Kunden nutzbar sein, die wenig Erfahrung mit IT-Technik haben.


\subsection{Musskriterien}

\begin{itemize}
  \item Kunden
    \begin{itemize}
      \item Formulare unter verschiedenen Kategorien suchen
      \item Formulare ausdrucken
    \end{itemize}
  \item Administratoren
    \begin{itemize}
      \item Formulare hinzufügen
      \item Formulare löschen
      \item Formulare für Frontend freischalten und Menüstruktur bearbeiten
      \item Farbschemata und Logos einstellen
      \item Einstellungen mitsamt aller Formulare exportieren und importieren 
    \end{itemize}
  \item System
    \begin{itemize}
        \item Stellt nach Start des Servers automatisch WLAN Access Point bereit und startet den Webserver und das Framework
        \item Clients greifen auf den Webserver zu
        \item Server generiert die passende Oberfläche zu den hinterlegten Formularstrukturen dynamisch 
    \end{itemize}{}
\end{itemize}

\newpage

\subsection{Wunschkriterien}

\begin{itemize}
  \item Kunden
    \begin{itemize}
        \item Formulare durch Einlesen ihrer eGK vorausfüllen
        \item Einen Videoanruf zur Beratung initiieren
    \end{itemize}
    \item Administratoren
        \begin{itemize}
            \item Synchronisation der Formulare auf mehreren Instanzen über eine temporäre Internetverbindung
        \end{itemize}
\end{itemize}
\vspace{1,5cm}
\subsection{Abgrenzungskriterien}

\begin{itemize}
  \item Das System eignet sich nur für einen Nutzer gleichzeitig.
  \item Das System wird ohne Internetzugang betrieben.
\end{itemize}
