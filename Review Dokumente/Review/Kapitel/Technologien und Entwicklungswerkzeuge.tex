\section{Technologien und Entwicklungsumgebung}
\subsection{Frontend}
Das Frontend wird mit HTML, CSS und JavaScript entwickelt. Für die Generierung von Seiten wir die Django Templatesprache verwendet.

\subsection{Backend}
Im Backend läuft ein Apache Webserver und das Django Webframework. Das Framework führt den Pythoncode aus, in dem die Modelle und Views geschrieben sind. Der Kleinstrechner stellt einen WLAN Access Point bereit, zu dem sich die Clients verbinden, um auf die Kunden- und Administratorseiten zuzugreifen.

\subsection{Entwicklungstools}
Als IDE verwenden wir Visual Studio Code. \\
Gitlab dient als Versionsverwaltung, mit dem eingebauten Wiki als Dokumentation und als Bugtracker. Zum Testen werden ein bereitgestelltes iPad, sowie ein Raspberry Pi genutzt. \\
Für das Pflichtenheft und die Reviewdokumente verwenden wir den Online-Latex-Editor Overleaf.