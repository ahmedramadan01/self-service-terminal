 \section{Bugreview}
 \subsection{Aufnahme von Bugs}
 Bugs werden vom Self-Service-Terminal Team im Bugtracker des zur Verfügung gestellten GitLabs angelegt. Der Bugtracker ist die zentrale Sammelstelle für alle gefundenen Bugs. Jedes Teammitglied hat Zugriff auf die Funktion und kann sich Bugs zuweisen, um sie anschließend zu bearbeiten. Dieser Vorgang wird manuell durchgeführt. Die entsprechend angelegten Issues werden mit einem eindeutig identifizierbaren Namen versehen. Außerdem kann den Issues ein Fälligkeitsdatum sowie ein korrespondierender Milestone zugewiesen werden. Abhängig von der Stelle an der ein Bug aufgetreten ist, wird er mit zugehörigen Tags versehen. Aktuell wird in den Tags zwischen \glqq Frontend\grqq{} und \glqq Backend\grqq{} unterschieden. Ein Issue kann das \glqq Frontend\grqq{}, \glqq Backend\grqq{} und beide Tags gleichzeitig zugeordnet bekommen.
 \vspace{1cm}
 \subsection{Priorisierung}
 Um gefundene Bugs in einer zielführenden Reihenfolge abzuarbeiten werden diese im Bugtracker durch Tags priorisiert. Die Priorisierung wird in einer Weise durchgeführt, sodass deutlich erkennbar ist wann ein Bug behoben sein muss um die gewünschte Funktionalität der Anwendung zu den gesetzten Deadlines zu erreichen. Die Deadlines werden hierbei durch eine Verknüpfung der Issues mit den Meilensteinen \glqq 2. Review\grqq{} und \glqq Testphase\grqq{} repräsentiert. Es wird außerdem zwischen der Schwere der Bugs unterschieden. Bugs, welche die Anwendung in der Erfüllung der Musskriterien hindert, werden mit dem Tag \glqq critical\grqq{} versehen. Diese Bugs haben gemessen an der zugehörigen Deadline die höchste Priorität.
 \newpage
 \subsection{Liste aufgenommener Bugs}
\begin{itemize}
    \item \textbf{/B001/} | \textcolor{violet}{Frontend} | \textcolor{cyan}{2. Review} | \\ \noindent \textit{\textbf{Unterscheidung von Formularen und Submenus in angezeigten Menus:}} \\ \noindent Formulare und Untermenus innerhalb von Menus lassen sich nur durch den angezeigten Text unterscheiden. Eine grafische Unterscheidungsmöglichkeit ist notwendig.
    \item \textbf{/B002/} | \textcolor{violet}{Backend} | \textcolor{cyan}{Testphase} | \\ \noindent \textit{\textbf{Hilfe Template bei nicht gesetzter Startseite anzeigen:}} \\ \noindent Die Startseite fungiert als Wurzel der Baumstruktur von Menus und Formularen. Ist die Startseite nicht gesetzt, kann die Menüstruktur nicht erzeugt werden. Ist keine Startseite gesetzt, sollte ein Hilfetext angezeigt werden.
    \item \textbf{/B003/} | \textcolor{violet}{Backend} | \textcolor{cyan}{2. Review} | \\ \noindent \textit{\textbf{Usability im Admin-Panel:}} \\ \noindent Wurde im Admin-Menü ein \textit{Terminal\_Settings} ForeignKey festgelegt, soll er für alle weiteren Untermenüs voreingestellt sein.
    \item \textbf{/B004/} | \textcolor{violet}{Backend/Frontend} | \textcolor{cyan}{2. Review} | \textcolor{red}{critical}\\ \noindent \textit{\textbf{Formular Vorschaubild anzeigen:}} \\ \noindent Bei Auswahl eines Formulars in einem Menü soll ein Vorschaubild des ausgewählten Formulars angezeigt werden.
    \item \textbf{/B005/} | \textcolor{violet}{Backend} | \textcolor{cyan}{Testphase} | \textcolor{red}{critical}\\ \noindent \textit{\textbf{Personendaten der Adminaccounts verschlüsseln:}} \\ \noindent Die Datenbankeinträge der Nutzeraccounts für Administratorenaccounts sind nicht verschlüsselt. Aus Sicherheitsgründen müssen die entsprechenden Daten verschlüselt werden.
    \item \textbf{/B006/} | \textcolor{violet}{Backend} | \textcolor{cyan}{Testphase} | \\ \noindent \textit{\textbf{Usability der Import/Export Funktion:}} \\ \noindent Die Datenbankeinträge für Menüs, Nutzer und Formulare müssen aktuell einzeln exportiert und importiert werden. Menüs und Formulare sollen zusammen exportierbar und importierbar sein.
\end{itemize}