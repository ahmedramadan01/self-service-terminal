\section{Ausblick auf die dritte Iteration}
\subsection{Testen des Systems}
Hauptaufgabe während der dritten Iteration ist das Testen des Self-Service-Terminal. Wir werden dafür automatische Tests definieren, um die implementierten Anforderungen ausführlich zu testen. Gefundene Bugs werden gemäß unserer Problemlösungsverfahren an Teammitglieder deligiert und behoben. Ziel ist es zum Ende der dritten Iteration alle Musskriterien fehlerfrei implementiert zu haben. Die zweite Hauptaufgabe während der Testphase besteht darin, das User-Interface auf Bedienbarkeit zu prüfen und gegebenenfalls Anpassungen vorzunehmen.\par
\vspace{0,5cm}
\subsection{Konfiguration der Hardware}
Das Self-Service-Terminal soll auf einem Raspberry Pi zum Einsatz kommen. Um die Randbedingungen für den Betrieb der Anwendung zu erfüllen, muss der Pi entsprechend konfiguriert werden. Diese Konfiguration soll nach Möglichkeit automatisch erfolgen, wenn die Anwendung installiert wird. Dazu wird ein Installationsskript nötig sein, dass bei der Installation ausgeführt wird. Dieses Skript wird in der 3. Iteration entwickelt und getestet. \par
\vspace{0,5cm}
\subsection{Deployment}
Das Self-Service-Terminal muss zum Abschluss der Entwicklung von der Entwicklungsumgebung in die Produktivumgebung überführt werden. Dazu ist es notwendig, vom Django-Entwicklungsserver auf einen Apache Webserver mit dem \textit{mod\_wsgi} Modul zu wechseln. Es muss außerdem eine Lösung entwickelt werden, das System möglichst benutzerfreundlich aufzusetzen.
