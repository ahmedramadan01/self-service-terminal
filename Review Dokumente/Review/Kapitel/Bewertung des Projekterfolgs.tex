\section{Bewertung des Projekterfolgs}

Ziel des Softwareprojekts: \glqq Self-Service-Terminal for Health Insurance Offices\grqq{} war es eine Webanwendung zu entwickeln, welche es Kunden erlaubt sich in Krankenkassenfilialen selbst zu bedienen. Das zu entwickelnde System ist als ein \glqq Proof of Concept\grqq{} zu verstehen. Ziel ist es, das SST in den realen Bedingungen einer Filiale zu testen um Erkenntnisse darüber zu gewinnen, inwiefern eine solche Lösung für Kunden attraktiv ist. Es ist dem Projektteam gelungen dieses Ziel umzusetzen. Sämtliche aus den Musskriterien abgeleiteten Anforderungen wurden implementiert und nachgewiesen. Dies ist die Mindestanforderung um die Entwicklung des SST als Erfolg anzusehen.\par
\noindent Zusätzlich zu den Musskriterien wurden von den Auftraggebern auch einige Wunschkriterien formuliert, die es nach Möglichkeit umzusetzen galt. Es ist uns nicht gelungen diese Kriterien rechtzeitig zu implementieren. Im Verlauf der dritten Iteration wurde vom Projektteam entschieden, von der Entwicklung besagter Anforderungen abzusehen, da die Musskriterien sonst nicht ausreichend getestet und kritische Bugs nicht hätten behoben werden können. Im Ergebnis heißt das, dass nicht alle von den Auftraggebern gestellten Anforderungen erfüllt werden konnten\par
\noindent Ziel des Softwareprojekts ist nicht nur die Entwicklung von Software, sondern auch der Lernerfolg der an der Projektarbeit beteiligten Studierenden. Durch das Projekt sollen System- Methoden- und Fachkompetenzen zur Entwicklung von Softwaresystemen erlernt, beziehungsweise vertieft werden. Diese Zielstellungen betrachten wir als erreicht. Zu Erwartungen des Teams zu Beginn des Projektes können retrospektiv nur als naiv bezeichnet werden. Mit jeder Iteration begegneten uns neue Herausforderungen, mit denen wir im Voraus nicht gerechnet hatten. Die Probleme erstreckten sich von einer realistischen Aufwandsschätzung über Kommunikation im Team bis zu Fachkenntnissen, welche zur Entwicklung von Webanwendungen notwendig sind. Diese Probleme konnten insoweit überwunden werden, dass am Ende eine lauffähige Webapplikation entstanden ist. Gleichzeitig haben sie aber genug Zeit gekostet um nicht alle Anforderungen erfüllen zu können. Der größte Erfolg in diesem Zusammenhang ist, das die nötigen Erfahrungen gesammelt wurden um entsprechende Fehler zukünftig nicht zu wiederholen.\par
\noindent Als Fazit kommen wir als Team zu dem Schluss, dass wir, wenn auch nicht vollständig zufrieden, aber dennoch der Meinung sind, die gestellten Aufgaben erfolgreich abgeschlossen zu haben. 