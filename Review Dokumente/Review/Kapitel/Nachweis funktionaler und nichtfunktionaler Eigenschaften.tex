\section{Nachweis funktionaler und nichtfunktionaler Eigenschaften}


\subsection{Benutzerfunktionen}

\begin{itemize}
    \item \textbf{/F0010/}\textit{Formulare auswählen:} \par
    Ein Benutzer kann Formulare in den Menüs suchen und sich ein Formular anzeigen lassen.\par
    \noindent \textbf{Test /T0010/:} Eingestellte Menüs sind aufrufbar, eingestellte Formulare sind aufrufbar. Es existiert eine Baumstruktur. Die Baumstruktur ist navigierbar.
    
    \item \textbf{/F0020/}\textit{Formulare ausdrucken:} \par
    Ein Benutzer kann ein gefundenes Formular über eine Schaltfläche in der Anwendung ausdrucken.\par
    \noindent \textbf{Test /T0020/:} Bei Betätigung des Drucken-Buttons wird der lpr-Befehl gesendet und http-Status-Code 204 zurückgegeben.\par
    \noindent \textbf{Test /T0030/:} Ist eine Formularseite im Frontend eingestellt, aber kein entsprechendes PDF hochgeladen, wird beim Aufruf der Drucken-Funktion eine entsprechende Fehlermeldung angezeigt.
\end{itemize}

\vspace{1,5cm}

\subsection{Administratorfunktionen}

    \begin{itemize}
        \item \textbf{/F0110/}\textit{Einfügen von Formularen:} \par
        Ein Administrator kann Formulare im Format PDF in das Backend hochladen und speichern.\par
        \noindent \textbf{Test /0040/:} Ein hochgeladenes PDF wird im Files-Ordner gespeichert, die zugehörige URL ist abrufbar, der Pfad zu dieser PDF ist in der Datenbank hinterlegt.
        
        \item \textbf{/F0120/}\textit{Bearbeiten von Formularen:} \par
        Ein Administrator kann die Metadaten eines hochgeladenen und gespeicherten Formulars, wie Titel, Beschreibung oder Position in der Menüstruktur, direkt auf der Admin-Seite bearbeiten.\par
        \noindent \textbf{Test /0050/:} Wird ein Formular gespeichert, werden alle Werte des Objekts der zugehörigen Relation in der Datenbank belegt. Die Werte sind aus der Admin-Seite änderbar.
        
        \item \textbf{/F0130/}\textit{Löschen von Formularen:} \par
        Ein Administrator kann ein auf der Admin-Seite gespeichertes Formular aus dem System löschen.\par
        \noindent \textbf{Test /0060/:} Datenbankeinträge zu einem Formular werden nach der Löschung des Formulars gelöscht. Fremdschlüsseleinträge zu entsprechendem Formular werden gelöscht.
        
        \item \textbf{/F0140/}\textit{Setzen der Farben und Logos:} \par 
        Ein Administrator kann die Farben des Frontends anpassen. Außerdem kann er Logos im jpg-Format hochladen und im Frontend anzeigen lassen.
        \noindent \textbf{Test /0070/:} Änderungen der Farben auf der Admin-Seiten führen zu Änderungen der entsprechenden Datenbankeinträge. Die geänderten Datenbankeinträge führen zu geänderten Farben im Frontend. Hochgeladene Logos werden gespeichert, Datenbankeinträge hierzu werden angelegt. Logos lassen sich im Frontend anzeigen.
        
        \item \textbf{/F0150/}\textit{Export und Import:} \par
        Die Einstellungen können als JSON Datei exportiert und importiert werden. Die Formulare werden zusammen mit der JSON-Datei in einer Ordnerstruktur exportiert.
        \noindent \textbf{Test /0080/:} Aufruf der Export-Funktion resultiert in der Erstellung einer JSON-Datei, in der die exportieren Werte gespeichert sind. Die Export-Datei ist im Files-Ordner zu finden. Import eine kompatiblen JSON-Datei führt zu den erwarteten Änderungen in der Datenbank.
    \end{itemize}{}
  
  \vspace{1,5cm}  
  \subsection{Systemfunktionen}
  
    \begin{itemize}
        \item \textbf{/F0250/}\textit{Dynamische Generierung von Seiten:} \par
        Der Server generiert dynamisch die passende Oberfläche zu den hinterlegten Formularstrukturen.
        \noindent \textbf{Test /0090/:} Werden mehr als fünf Untermenüs oder Formulare einem Menü zugeordnet, werden automatisch weitere Seiten zur Anzeige der Formulare und Untermenüs erzeugt. Diese Seiten sind vom ursprünglichen Menü aus aufrufbar.
    \end{itemize}