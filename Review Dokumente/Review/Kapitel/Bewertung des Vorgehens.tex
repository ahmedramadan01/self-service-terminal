\section{Bewertung des Vorgehens}

Als Vorgehenmodell haben wir zu Beginn der ersten Iteration eine agile Variation des Unified Process gewählt. Dieses Vorgehensmodell ist in diesem Dokument auf Seite 4 unter dem Punkt  \glqq Vorgehen\grqq{} beschrieben. Die Wahl eines agilen Vorgehensmodells bewerten wir auch gegen Ende des Projekts als positiv. Unser Team, sowie die einzelnen Teammitglieder, vor Abschluss des Projekts wenig oder keine Erfahrung mit der Entwicklung von Software. Gleiches gilt für die Arbeit mit dem Django Framework. Unter diesen Voraussetzungen wäre es kaum machbar gewesen eine Vorplanung, wie beispielweise im klassischen Wasserfallmodell gefordert, mit zielführenden Ergebnissen abzuschließen. \par
\noindent Während der Entwicklung des Self-Service-Terminals wurden wir pausenlos mit Problemen konfrontiert, die wir aufgrund unserer mangelnden Erfahrung nicht vorausgesehen hatten. Durch die Wahl des agilen Vorgehensmodells, konnten wir uns schnell an entsprechend neue Gegebenheiten anpassen. Das iterative Vorgehen hat es uns ermöglicht, uns die notwendigen Kenntnisse zur Entwickung mit Django während der Arbeit anzueignen, indem wir Anforderungen einzeln implementiert haben. Durch konservative Zielsetzungen bei der Planung der zu entwickelnden Prototypen, war es uns möglich mit der Zeit ein erfolgversprechendes Softwarekonzept zu entwickeln.\par
\noindent Die agile Variante des Unified Process unterscheidet fünf Phasen, welche in jeder Iteration durchlaufen werden sollen. Diese Phasen sind: \glqq Planung\grqq{}, \glqq Analyse\grqq{}, \glqq Design\grqq{}, \glqq Implementierung\grqq{} und, \glqq Test\grqq{}. Probleme ergaben sich für uns hauptsächlich in den Phasen \glqq Analyse\grqq{} und \glqq Test\grqq{}. Nachdem die Anforderungen für einen Prototypen geplant wurden, begann die Implementiertung meist direkt. Das Überspringen der Analyse bedeutete, dass einige Implementierungen während der dritten Iteration angepasst werden mussten. Die Funktionen waren nicht dazu geeignet im Gesamtsystem nach Spezifikation zu funktionieren. Auch Tests kamen während der Entwicklung der Prototypen zu kurz. Dies führte zu einer erhöhten Anzahl an zu behebenden Bugs in der dritten Iteration.\par
\noindent Insgesamt lässt sich durch diese Versäumnisse erklären, warum nicht genug Zeit blieb auch noch Wunschkriterien zu implementieren. Insgesamt sind wir aber der Meinung, dass wertvolle Erfahrungen für zukünftige Arbeit mit agilen Vorgehensmodellen gesammelt wurden. Wir sind zuverlässig gemachte Fehler in Zukunft nicht mehr zu wiederholen.

