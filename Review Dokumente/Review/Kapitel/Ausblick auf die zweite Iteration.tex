\section{Ausblick auf die zweite Iteration}
Das Ziel der zweiten Iteration besteht in der Implementierung aller Musskriterien. Dabei sollen bereits Testfälle definiert und implementiert werden. Optional soll an den Wunschkriterien gearbeitet werden. Hier sollen ebenso Tests implementiert werden. Die Arbeitsweise ist wie folgt: \\

\noindent Alle Anforderungen werden als Issues in Gitlab eingetragen und können einzelnen Teammitgliedern zugewiesen werden. Der aktuelle Stand wird im Kommentarbereich des Issues dokumentiert. Wenn eine Anforderung fertiggestellt und getestet worden ist, wird das Issue geschlossen. Die Anforderungen sind vergleichbar mit den "`user stories"', die in der agilen Entwicklung verwendet werden. Alle zwei Wochen trifft sich die Große Runde bestehend aus allen Teammitgliedern, den Betreuern und Auftraggebern und tauscht sich über den Fortschritt des Projekts aus. Es wird der aktuelle Prototyp vorgeführt. \\

\noindent Am Ende der zweiten Iteration soll als Ergebnis ein Prototyp stehen, der alle Musskriterien erfüllt. Außerdem soll der erstellte Code dokumentiert worden sein. Ein Model-Template-View Diagramm stellt die innere Struktur der Software dar. 