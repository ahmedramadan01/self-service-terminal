\section{Problemlösung}
Die Entwicklung des Self-Service-Terminal erfolgt iterativ bezogen auf die zu erfüllenden Musskriterien. Dazu werden im zentralen GitLab Issues erstellt und einzelnen Teammitgliedern zugewiesen. Die jeweils zugewiesenen Teammitglieder arbeiten sich selbstständig in die Aufgabenstellung ein und versuchen eine Lösung des ausgewählten Implementierungsproblems zu finden. Kommt es bei der Implementierung der Funktionen im gesetzten Zeitraumn zu Problemen, besteht die Möglichkeit sich an das Team zu wenden und die Probleme gemeinsam anzugehen. Die Funktionen der Anwendung werden entweder in einem aufgabenbezogenen Branch oder direkt im \textit{develop} Branch erstellt. Die Implementierung einer Funktion im Rahmen des zweiten Reviews ist abgeschlossen, wenn die im Pflichtenheft spezifizierte Anforderung erfüllt ist.

\subsection{Implementierungsentscheidungen}
Die Entscheidung, wie genau eine Anforderung zu implementieren ist, liegt bei dem Teammitglied, welchem das entsprechende Issue zugewiesen ist. Wir orientieren uns dabei wie in unseren Werten und Prinzipien beschrieben am KISS-Prinzip: \glqq Keep it short an simple\grqq{}. Das bedeutet, dass immer eine möglichst einfache Lösung des bestehenden Implementierungsproblems zu bevorzugen ist. Die Entwicklung während der zweiten Iteration zielt darauf ab, die Grundfunktionalitäten des Self-Service-Terminal zu implementieren. Performanz und Speicherausnutzung der gewählten Lösung spielen während der zweiten Iteration deshalb eine untergeordnete Rolle. Während der Testphase des Softwareprojekts können die Lösungen dann weiterentwickelt werden, sollte dies notwendig sein. Sämtliche entwickelten Lösungen 