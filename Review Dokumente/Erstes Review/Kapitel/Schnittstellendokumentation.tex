\section{Schnittstellendokumentation}
Das Self-Service-Terminal benötigt zum Abschluss der zweiten Iteration an drei Stellen Zugriff auf Schnittstellen. Einmal sind einige Schnittstellen innerhalb der Systemgrenzen notwendig um die Grundfunktionalität des Django Frameworks sicherzustellen. Diese Schnittstellen bringt das Framework mit, sie werden automatisch genutzt, wenn in Django nach dem MTV-Prinzip gearbeitet wird. Darüber hinaus kommuniziert das Selt-Service-Terminal noch mit der CUPS-Druckersoftware und dem Apache Webserver über das Apache Modul mod\_wsgi (Python: Web Server Gateway Interface).
\subsection{Django}
Innerhalb der Systemgrenzen des SST werden ausschließlich die Schnittstellen verwendet, welche Django aufgrund seines Aufbaus mitbringt. Der Verwendung dieser Schnittstellen ergibt sich automatisch, wenn eine Webanwendung nach dem Model-Template-View-Prinzip entwickelt wird. Als Beispiel möchten wir kurz die sogenannten \glqq Manager\grqq{} anführen. Als Manager werden in Django die Schnittstellen bezeichnet, durch welche Datenbank-Anfrage-Operationen für die Modelle realisiert werden. Standardmäßig wird für jedes Modell eine solche Manger-Klasse angelegt. Details zur Verwendung von Schnittstellen im Django Framework lassen sich der Django\footnote{\href{https://docs.djangoproject.com/en/3.0/}{https://docs.djangoproject.com/en/3.0/}, abgerufen am 09.06.2020} Dokumentation entnehmen.

\subsection{CUPS}
Das Self-Service-Terminal verwendet die CUPS-Druckersoftware um die Druckfunktionalität für ausgewählte Formulare über den Raspberry Pi umzusetzen. Dazu wird ein Python-Subprocess aufgerufen, der die Kommandozeile aufruft. Der Kommandozeile wird der Befehl: lpr \textit{Dateiname} übergeben. Die Rückgabe der Kommandozeile wird als Objekt zurück an das SST geschickt. Das Objekt enthält die Standard-Datenströme \textit{stderr}, \textit{stdout} und der Rückgabewert des \textit{lpr}-Befehls. Aufgrund dieser Informationen wird der Druckvorgang entweder bestätigt, oder eine Fehlermeldung ausgegeben. 
\subsection{mod\_wsgi}
Um die Kommunikation einer Python-Anwendung mit dem Apache Webserver zu gewährleisten ist es erforderlich Apache um das Modul \textit{mod\_wsgi} zu erweitern. WSGI ist dabei der Standard für Python Webapplikationen. Genauere Informationen zur Funktionsweise des Moduls sind in der \footnote{\href{https://modwsgi.readthedocs.io/en/develop/index.html}{https://modwsgi.readthedocs.io/en/develop/index.html}, abgerufen am 09.06.2020} zu finden.
