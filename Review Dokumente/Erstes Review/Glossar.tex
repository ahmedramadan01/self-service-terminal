\section{Glossar}

\begin{description}
  \item[Back-End]
    Nur für Administratoren zugängliche Seiten zur Systemverwaltung, d.h. Einpflegen, Speichern, Bearbeiten, Exportieren und Löschen von Formularen sowie das Einstellen von Farbschemata und Logos im Front-End.
 
  \item[Django]
    ist ein in Python geschriebenes, quelloffenes Webframework, das einem Model-Template-View-Schema folgt.
    \footnote{\href{https://docs.djangoproject.com/en/dev/faq/general}{docs.djangoproject.com/en/dev/faq/general, abgerufen am 08.05.2020}}

 \item[Elektronische Gesundheitskarte]
    Chipkarte im Scheckkartenformat, die als erweiterte Versichertenkarte für gesetzlich Krankenversicherte fungiert.
    \footnote{\href{https://de.wikipedia.org/w/index.php?title=Elektronische_Gesundheitskarte&oldid=198562236}{de.wikipedia.org/w/index.php?title=Elektronische\_Gesundheitskarte,  abgerufen am 08.05.2020}}
    \footnote{\href{https://www.gesetze-im-internet.de/sgb_5/__291.html}{www.gesetze-im-internet.de/sgb\_5/\_\_291.html, abgerufen am 08.05.2020}}
 
  \item[Front-End]
    Webseite für die Kundenoberfläche.
    
  \item[Git]
    ist eine freie Software zur verteilten Versionsverwaltung von Dateien.
    \footnote{\href{https://de.wikipedia.org/w/index.php?title=Git&oldid=199354247}{de.wikipedia.org/w/index.php?title=Git, abgerufen am 09.05.2020}}
    
  \item[GitLab]
    ist eine Webanwendung zur Versionsverwaltung für Softwareprojekte auf Basis von Git.
    \footnote{\href{https://de.wikipedia.org/w/index.php?title=GitLab&oldid=196028257}{de.wikipedia.org/w/index.php?title=GitLab, abgerufen am 09.05.2020}}
    
  \item[KISS]
    "Keep it small and simple" bzw "Keep it simple, stupid". Ist ein Prinzip der agilen Vorgehensweise welches besagt, dass zu einem Problem eine möglichst einfache Lösung gefunden werden soll.

  \item[iPad]
    Tabletcomputer des Herstellers Apple Inc.

  \item[Model Layer]
    Eine Abstraktionsschicht in Django, die zur Strukturierung und Manipulation der Daten der Webapplikation dient.
    \footnote{\href{https://docs.djangoproject.com/en/3.0/}{docs.djangoproject.com/en/3.0/, abgerufen am 08.05.2020}}
 
  \item[Model-Template-View]
    Djangos Umsetzung des Model-View-Controller Musters.

  \item[Model-View-Controller]
    Model View Controller ist ein Muster zur Unterteilung einer Software in die drei Komponenten: Datenmodell (model), Präsentation (view) und Programmsteuerung (controller).
    \footnote{\href{https://de.wikipedia.org/w/index.php?title=Model_View_Controller&oldid=195305891}{de.wikipedia.org/w/index.php?title=Model\_View\_Controller, abgerufen am 08.05.2020}}
    
  \item[Overleaf]    
    ist ein Online LaTeX Editor. 

  \item[Python]
    ist eine universelle, interpretierte, multiparadigmatische Programmiersprache.
 
  \item[Views]
    Kapseln in Django die Logik, welche für die Verarbeitung der Anfrage eines Nutzer und die Rückgabe einer Antwort verantwortlich ist.

  \item[Raspbian]
    Linux-Betriebssystem auf Basis von Debian. Wird hauptsächlich auf den Raspberry Pi Einplatinencomputern eingesetzt.
    
  \item[RocketChat]
    ist eine freie und quelloffene Chatsoftware.

  \item[Self-Service-Terminal]
    Ein Computer, der von Kunden in einem Geschäft oder einer anderen Einrichtung zur Selbstbedienung genutzt werden kann. Oftmals mit eingebautem Touchscreen.
\end{description}
