\section{Produktfunktionen}


\subsection{Benutzerfunktionen}

\begin{itemize}
    \item \textbf{/F0010/}\textit{Formulare auswählen:} \par
    Ein Benutzer kann Formulare in den Menüs suchen und sich ein Formular anzeigen lassen.
    \item \textbf{/F0020/}\textit{Formulare ausdrucken:} \par
    Ein Benutzer kann ein gefundenes Formular über eine Schaltfläche in der Anwendung ausdrucken.
\end{itemize}
\vspace{1,5cm}
\textbf{Optional:}
    \begin{itemize}
        \item \textbf{/F0030/}\textit{Schnellsuche:} \par
        Der Benutzer kann durch Eingeben eines Suchbegriffs in eine Suchleiste den in der Datenbank hinterlegten Namen eines Formulars direkt suchen. Ist der Name hinterlegt, wird das entsprechende Formular zur Auswahl angezeigt.
        \item \textbf{/F0040/}\textit{eGK einlesen und Formulare vorausfüllen:} \par
        Wurde vom Benutzer ein Formular ausgewählt, kann er seine eGK an einem am IPad angeschlossenen Kartenlesegerät einlesen lassen. Das System füllt das ausgewählte Formular mit den ausgelesenen Daten des Benutzers.
        \item \textbf{/F0050/}\textit{persönliche Beratung durch Videoanruf:} \par
        Der Kunde kann an beliebiger Stelle des Prozesses durch einen Tastendruck einen Videoanruf mit einem Mitarbeiter der Filiale starten.
    \end{itemize}
    
\newpage

\subsection{Administratorfunktionen}

    \begin{itemize}
        \item \textbf{/F0110/}\textit{Einfügen von Formularen:} \par
        Ein Administrator kann Formulare im Format PDF in das Backend hochladen und speichern.
        
        \item \textbf{/F0120/}\textit{Bearbeiten von Formularen:} \par
        Ein Administrator kann die Metadaten eines hochgeladenen und gespeicherten Formulars, wie Titel, Beschreibung oder Position in der Menüstruktur, direkt im Backend bearbeiten.
        
        \item \textbf{/F0130/}\textit{Löschen von Formularen:} \par
        Ein Administrator kann ein im Backend gespeichertes Formular aus dem Backend löschen.
        
        \item \textbf{/F0140/}\textit{Setzen der Farben und Logos:} \par 
        Ein Administrator kann die Farben des Frontends anpassen. Außerdem kann er Logos im jpg-Format hochladen und im Frontend anzeigen lassen.
        
        \item \textbf{/F0150/}\textit{Export und Import:} \par
        Die Einstellungen können als JSON Datei exportiert und importiert werden. Die Formulare werden zusammen mit der JSON-Datei in einer Ordnerstruktur exportiert.
    \end{itemize}{}
\vspace{1,5cm}
\textbf{Optional:}
    \begin{itemize}
        \item \textbf{/F0160/}\textit{Updates:} \par 
        Ein Administrator kann über eine zeitlich begrenzte Internetverbindung das System updaten.
        
        \item \textbf{/F0170/}\textit{Onlinekonfigurationsabgleich:} \par
        Ein Administrator kann den Abgleich der Einstellungen mit einem zentralen Server der Filiale starten.
    \end{itemize}
    
\newpage

\subsection{Systemfunktionen}

    \begin{itemize}
        \item \textbf{/F0210/}\textit{Offlinefähigkeit:} \par
        Das System, bestehend aus Server, Clients und Drucker, kann ohne aktive Internetverbindung genutzt werden.
        
        \item \textbf{/F0220/}\textit{WLAN:} \par
        Der Server stellt einen WLAN Access Point für die Clients und den Drucker bereit.
        
        \item \textbf{/F0230/}\textit{Ausfallsicherheit:} \par
        Bei einem Start / Neustart startet der Server Access Point, Webserver und Framework automatisch, sodass volle Funktionalität hergestellt ist.
        
        \item \textbf{/F0240/}\textit{Webserver:} \par
        Der Server stellt einen Webserver bereit, auf den die Clients zugreifen.
        
        \item \textbf{/F0250/}\textit{Dynamische Generierung von Seiten:} \par
        Der Server generiert dynamisch die passende Oberfläche zu den hinterlegten Formularstrukturen.
    \end{itemize}