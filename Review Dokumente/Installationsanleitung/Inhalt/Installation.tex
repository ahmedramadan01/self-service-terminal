\section{Installation}
Nachdem die Vorbereitungen abgeschlossen wurden, können Sie nun mit der Installation der Anwendung beginnen. Die Befehle zur Installation des Self-Service-Terminals sollen:

\begin{itemize}
    \item im Virtual Environment
    \item mit root-Rechten
    \item im Installationsverzeichnis
\end{itemize}

\noindent ausgeführt werden.\\

\noindent Bringen Sie zunächst ihr System auf den neusten Stand und installieren Sie danach alle benötigten Pakete:\\

\textbf{’sudo apt update \&\& sudo apt upgrade -y’}\par
\textbf{’sudo apt install apache2 apache2-dev cups hostapd dnsmasq’}\\

\noindent Anschließend installieren Sie alle benötigten pip-Pakte:\\

\textbf{’pip install django django-import-export mod\_wsgi Pillow pdf2image’}\\

\noindent Nachdem alle benötigten Pakete installiert worden sind, klonen Sie den Source-Code des Self-Service-Terminal in Ihr Installationsverzeichnis:\\

\textbf{’git clone url/zum/Self-Service-Terminal’}\\

\noindent Um die Funktionalität des Self-Service-Terminal zu gewährleisten, muss nun noch die Software für den Wlan-Access-Point sowie den Webserver und konfiguriert werden.