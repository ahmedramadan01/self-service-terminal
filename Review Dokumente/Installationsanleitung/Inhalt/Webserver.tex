\section{Webserver}
Um das Self-Service-Terminal mit dem Apache Webserver zu betreiben, muss Apache um das Modul mod\_wsgi erweitert werden. Dazu muss das Modul installiert und der Output des Installationsbefehls in eine Konfigurationsdatei kopiert werden. Führen Sie zur Installation des Moduls folgenden Befehl aus:\\

\textbf{sudo mod\_wsgi-express install-module}\\

\noindent Den Output dieses Befehls kopieren Sie in folgende Datei:\\

\textbf{'/etc/apache2/mods-available/wsgi.load'}\\

\noindent Anschließend aktivieren Sie das Modul mit folgendem Befehl:\\

\textbf{sudo a2enmod wsgi}\\

\noindent Als Nächstes muss die Standardkonfiguration des VirtualHost angepasst werden. Dazu wird folgende Datei angepasst:\\

\textbf{/etc/apache2/sites-available/000-default.conf}\\

\noindent Die Konfigurationsdatei muss um den Pfad zum Virtual Environment \textcolor{red}{\{venv\_dir\}} (z.B.: /home/user/.venv/Self\_Service\_Terminal) und den Pfad zum Installationsverzeichnis des Self-Service-Terminal \textcolor{blue}{\{sst\_dir\}} ergänzt werden. Es folgt der Inhalt der Konfigurationsdatei. Die Stellen, an denen die Pfade ergänzt werden müssen, sind farblich markiert.\\

<VirtualHost *:80>\par
	\# The ServerName directive sets the request scheme, hostname and port that\par
	\# the server uses to identify itself. This is used when creating\par 
	\# redirection URLs. In the context of virtual hosts, the ServerName\par
	\# specifies what hostname must appear in the request's Host: header to\par
	\# match this virtual host. For the default virtual host (this file) this\par
	\# value is not decisive as it is used as a last resort host regardless.\par
	\# However, you must set it for any further virtual host explicitly.\par
	\#ServerName www.example.com\\

	ServerAdmin webmaster@localhost\par
	DocumentRoot /var/www/html\\
	
	\newpage
	
	\# Available loglevels: trace8, ..., trace1, debug, info, notice, warn,\par
	\# error, crit, alert, emerg.\par
	\# It is also possible to configure the loglevel for particular\par
	\# modules, e.g.\par
	\#LogLevel info ssl:warn\\

	ErrorLog \$\{APACHE\_LOG\_DIR\}/error.log\par
	CustomLog \$\{APACHE\_LOG\_DIR\}/access.log combined\\

	\# For most configuration files from conf-available/, which are\par
	\# enabled or disabled at a global level, it is possible to\par
	\# include a line for only one particular virtual host. For example the\par
	\# following line enables the CGI configuration for this host only\par
	\# after it has been globally disabled with "a2disconf".\par
	\#Include conf-available/serve-cgi-bin.conf\\
	
	WSGIDaemonProcess sp4 python-home=\textcolor{red}{\{venv\_dir\}}\par python-path=\textcolor{blue}{\{sst\_dir\}}/softwareprojekt---self-service-terminal/sp4\par
	WSGIProcessGroup sp4 \par
	WSGIScriptAlias / \textcolor{blue}{\{sst\_dir\}}/softwareprojekt---self-service-terminal/sp4/sp4/wsgi.py\\
	
	Alias /files/ \textcolor{blue}{\{sst\_dir\}}/softwareprojekt---self-service-terminal/sp4/self\_service\_terminal/files/\par
	Alias /static/ \textcolor{blue}{\{sst\_dir\}}/softwareprojekt---self-service-terminal/sp4/self\_service\_terminal/static/\\

	<Directory \textcolor{blue}{\{sst\_dir\}}/softwareprojekt---self-service-terminal/sp4/self\_service\_terminal/static>\par
	Require all granted\par
	</Directory>\\

	<Directory \textcolor{blue}{\{sst\_dir\}}/softwareprojekt---self-service-terminal/sp4/self\_service\_terminal/files>\par
	Require all granted\par
	</Directory>\\


	<Directory \textcolor{blue}{\{sst\_dir\}}/softwareprojekt---self-service-terminal/sp4/sp4>\par
	<Files wsgi.py>\par
	Require all granted\par
	</Files>\par
	</Directory>\\	

</VirtualHost>\\

\newpage

\noindent Als letztes müssen dem Webserver noch Schreibrechte für die Datenbank und auszuliefernden Dateien gegeben werden. Dazu führen Sie folgende Befehle im Installationsverzeichnis der Anwendung aus:\\

\textbf{sudo chown www-data:www-data database}\par
\textbf{sudo chown www-data:www-data database/db.sqlite3}\\

\textbf{mkdir files}\par
\textbf{mkdir files/forms}\par
\textbf{mkdir files/images}\par
\textbf{chown www-data:www-data files/}\\

\noindent Der Apache2 Webserver ist nun vollständig installiert und konfiguriert.