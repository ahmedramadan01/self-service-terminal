\section{Webserver}
Zur Installation des Webservers erstellen Sie folgende Datei:\\

\textbf{/lib/systemd/system/runmodwsgi.service}\\

\noindent Füllen Sie die Datei mit folgendem Inhalt:\\

\textbf{[Unit]}\par
\textbf{Description=Run mod\_wsgi}\par
\textbf{After=multi-user.target}\\

\textbf{[Service]}\par
\textbf{ExecStart=\{venv\_path\}/bin/python3 \{installation\_path\}/sp4/manage.py \indent runmodwsgi \texttt{-{}-}
reload-on-change \texttt{-{}-}user} \textbf{\{username\}}\par\textbf{\texttt{-{}-}group \{username\}}\\


\textbf{[Install]}\par
\textbf{WantedBy=multi-user.target}\\

\noindent Ersetzen Sie die Ausdrücke in geschwungenen Klammern \{\} mit folgenden Werten:\\
\noindent \textbf{\{venv\_path\}:} Absoluter Pfad zu Ihrem Python Virtual Environment.\par
\noindent \textbf{\{installation\_path\}:} Absoluter Installationspfad des Self-Service-Terminal\par
\noindent \textbf{\{username\}:} Name des Nutzers in dessen Home-Verzeichnis das Self-Service-Terminal installiert wurde.\\

\noindent Anschließend führen Sie folgende Befehle aus:\\

\textbf{’sudo systemctl daemon-reload’}\par
\textbf{’sudo systemctl enable runmodwsgi.service’}\\

\noindent Das Self-Service-Terminal ist nun nach jedem Neustart auf Port 8000 verfügbar.