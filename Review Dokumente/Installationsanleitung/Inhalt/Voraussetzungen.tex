\section{Voraussetzungen}
Zur Installation des Self-Service-Terminals müssen folgende Voraussetzungen erfüllt sein:
\begin{itemize}
    \item Debian basiertes Linux-Betriebssystem
    \item Internes oder externes WLAN-Modul
    \item Python Version 3.8 oder neuer
\end{itemize}

\section{Vorbereitung}
Wir empfehlen die Installation des Self-Service-Terminals in einem Python Virtual Environment durchzuführen. Zur Einrichtung und Aktivierung des Virtual Environments, folgen Sie folgender Anleitung:\\

\noindent Installieren Sie das python3-venv Paket:\\

\textbf{’sudo apt install python3-venv’}\\

\noindent Sobald das Paket installiert ist, verwenden Sie folgenden Befehl um das Virtual Environment in einem Verzeichnis ihrer Wahl zu erstellen:\\

\textbf{’python3 -m venv /pfad/zum/venv/venv\_name’}\\

\noindent Wechseln Sie nun in das Verzeichnes des Virtual Environments und führen sie das Aktivierungs-Skript mit folgendem Befehl aus:\\

\textbf{’source venv\_name/bin/activate’}\\

Die Einrichtung des Virtual Environments ist damit abgeschlossen.
Um die Vorbereitungen zur Installation des Self-Service-Terminal abzuschließen, erstellen sie das Verzeichnis in welches die Anwendung installiert werden soll:\\

\textbf{’mkdir /home/{user}/Self-Service-Terminal’}\\



