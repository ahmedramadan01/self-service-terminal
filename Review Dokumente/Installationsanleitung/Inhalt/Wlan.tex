\section{Wlan-Access-Point}

Das Self-Service-Terminal benötigt zum Betrieb ein Wlan-Netz, welches vom Server bereitgestellt wird. Die dafür benötigte Software wurde bereits im Schritt 3: \glqq Installation\grqq{} installiert. Nun muss diese Software noch konfiguriert werden.\par
\noindent Zuerst muss die Anwendung hostapd aktiviert werden, danach wird die Firewall für den Betrieb des Netzwerks vorbereitet. Um dies zu erreichen führen Sie folgende Befehle aus:\\

\textbf{’sudo systemctl unmask hostapd’}\par
\textbf{’sudo systemctl enable hostapd’}\par
\textbf{’sudo DEBIAN\_FRONTEND=noninteractive apt install -y netfilter-\indent persistent iptables-persistent’}\\

\noindent Als nächstes muss der DHCP-Server konfiguriert werden. Rufen Sie dafür folgende Konfigurationsdatei auf:\\

\textbf{/etc/dhcpcd.conf}\\

\noindent Ergänzen Sie diese Datei um folgenden Inhalt:\\

\textbf{interface wlan0}\par
\textbf{static ip\_address=192.168.4.1/24}\par
\textbf{nohook wpa\_supplicant}\\

\noindent Um das System zu updaten, ist es notwendig hin und wieder eine Internetverbindung herzustellen. Dies lösen Sie indem sie IP-Forwarding aktivieren. Durch diese Funktion, können alle Geräte, welche sich im Wlan-Netzwerk befinden, auf das Internet zugreifen, wird der Server mit dem Internet verbunden. Um dies zu erreichen, erstellen Sie folgende Datei:\\

\textbf{/etc/sysctl.d/routed-ap.conf}\\

\noindent In diese Datei schreiben Sie nun folgende Zeile:\\

\textbf{net.ipv4.ip\_forward=1}\\

\noindent Passen Sie nun die Firewall-Regeln entsprechend an, indem Sie folgende Befehle ausführen:\\

\textbf{’sudo iptables -t nat -A POSTROUTING -o eth0 -j}\par \textbf{MASQUERADE’}\par
\textbf{’sudo netfilter-persistent save’}\\

\newpage

\noindent Um das Netzwerk betreiben zu können, ist ein DNS-Server nötig. Diesen konfigurieren Sie, indem sie folgende Datei:\\

\textbf{/etc/dnsmasq.conf}\\

\noindent um einige Einstelungen ergänzen:\\

\textbf{interface=wlan0}\par
\textbf{dhcp-range=192.168.4.2,192.168.4.20,255.255.255.0,24h}\par
\textbf{domain=wlan}\par
\textbf{address=/gw.wlan/192.168.4.1}\\

\noindent Nachdem diese Einstellungen getroffen wurden, aktivieren Sie nun das Wlan-Modul des Servers mit folgendem Befehl:\\

\textbf{’sudo rfkill unblock wlan}’\\

\noindent Anschließend muss noch der Access-Point selbst konfiguriert werden. Sie finden die Konfigurationsdatei unter folgendem Pfad:\\

\textbf{/etc/hostapd/hostapd.conf}

\noindent Folgende Einstellungen sind zu treffen:\\

\textbf{country\_code=\{\}}\par
\textbf{interface=wlan0}\par
\textbf{ssid=\{\}}\par
\textbf{hw\_mode=g}\par
\textbf{channel=7}\par
\textbf{macaddr\_acl=0}\par
\textbf{auth\_algs=1}\par
\textbf{ignore\_broadcast\_ssid=0}\par
\textbf{wpa=2}\par
\textbf{wpa\_passphrase=\{\}}\par
\textbf{wpa\_key\_mgmt=WPA-PSK}\par
\textbf{wpa\_pairwise=TKIP}\par
\textbf{rsn\glqq\_pairwise=CCMP}\\

\noindent Diese Einstellungen müssen noch durch einige Informationen von Ihnen ergänzt werden. Als \textbf{country\_code=\{\}} wählen Sie \glqq DE\grqq{} indem Sie es zwischen die geschweiften Klammern schreiben. Bei \textbf{ssid=\{\}} verfahren Sie genauso. Sie wählen hier den gewünschten Namen des Wlan-Netzwerks. Schließlich wählen sie unter \textbf{wpa\_passphrase=\{\}} ein starkes Passwort für Ihr Netzwerk.\\

\noindent Um den Access Point verfügbar zu machen, müssen Sie nun nur noch das System mit folgendem Befehl neu starten:\\

\textbf{’sudo systemctl reboot’}\\

\noindent Für die Arbeit mit \textbf{hostapd} möchten wir auf folgende kurze Liste von nützlichen Befehlen hinweisen:\\

\begin{itemize}
    \item \textbf{’systemctl status hostapd’}
    \item \textbf{’sudo systemctl stop hostapd’}
    \item \textbf{’sudo systemctl start hostapd’}
    \item \textbf{’sudo systemctl disable hostapd’}
    \item \textbf{’sudo systemctl enable hostapd’}
\end{itemize}
